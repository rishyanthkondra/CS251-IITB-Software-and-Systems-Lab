\documentclass{beamer}
\usepackage{graphicx}
\usepackage{amsmath}
\usepackage{kpfonts}
\usepackage{boxedminipage}
\usepackage{bcprules}
\usetheme{CambridgeUS}

%%%%%%%%%%%%%%%%Title Page%%%%%%%%%%%%%%%%%%%%%%%%%%%%%%%%%

\title[Lambda Calculus]{Introduction to Beamer}
\subtitle[]{}
\author[F. last]{Firstname Lastname}
\institute[IITB]{
  Department of Computer Science and Engineering\\
  IIT Bombay.\\
  Powai, Mumbai - 400076\\[1ex]
  \texttt{userid@cse.iitb.ac.in}
}
\date[\today]{\today}

%%%%%%%%%%%%%%%%%%%%%%%%%%%%%%%%%%%%%%%%%%%%%%%%%%%%%%%%%%%
\newtheorem{exercise}{Exercise}
\begin{document}
%--- the titlepage frame -------------------------%
\begin{frame}[plain]
  \titlepage 
\end{frame}

%%%%%%%%%%%%%%%%%%%%%%%%%%%%%%%%%%%%%%%%%%%%%%%%%%%%%%%%%%%
\begin{frame}[fragile]{\bf  This is the title}
Beamer is a \LaTeX \:class for preparing presentations.

\begin{enumerate}
\item Slides are called frames in Beamer.
\item This is the usual ordered list in \LaTeX.
\item Following slides will contain random content which will show you various ways of using it. You need to replicate it.
\item Of course! we will give you boilerplate code!
\end{enumerate}
\end{frame}

\end{document}
